% Remove 't' to vertically align content within slides.
\documentclass[12pt, t]{beamer}

% Fonts are difficult with pdflatex. We're supposed to use Arial,
% but that pretty much requires XeLaTeX or similar. As Arial is
% really just a Helvetica clone and we do have access to Helvetica,
% we'll just roll with that instead. It looks "close enough" for
% most purposes.
\usepackage{helvet}

% In case you have Arial installed, you can probably comment out the
% above line, uncomment the two below lines and compile with XeLaTeX.
%\usepackage{fontspec}
%\setmainfont{Arial}

% Default to UTF8 input encoding: plenty of non-ASCII characters
% in Finnish names etc.
\usepackage[utf8]{inputenc}

% Defaults language to English.
\usepackage[english]{babel}

% Used for dummy text only, can be safely removed
% once \lipsum's are removed from body.
\usepackage{lipsum}

% Possible options (select 1):
%    general - General university theming
%    agri - Faculty of Agriculture and Forestry
%    arts - Faculty of Arts
%    behav - Faculty of Educational Sciences
%    bio - Faculty of Biological and Environmental Sciences
%    law - Faculty of Law
%    med - Faculty of Medicine
%    pharm - Faculty of Pharmacy
%    sci - Faculty of Science
%    soc - Faculty of Social Sciences
%    sockom - Swedish School of Social Science
%    theo - Faculty of Theology
%    vet - Faculty of Veterinary Medicine
% Using no options is the same as setting "general"
\usetheme[sci]{unihel}

% Title of your presentation
\title{presentation title that is super long and goes 2 lines}

% Subtitle of your presentation, optional
\subtitle{subtitle}

% Can be used to display funding info on the title slide. 
% Logo is optional. As is the whole thing.
\funding[eu-flag.jpg]{The So-and-So project has received funding from from some random party as grant n.o. xxxxxxx}

% Name in square brackets is used in footer
% Name in curly brackets is used on title page
\author[Leppänen et al.]{Leo Leppänen}

% Set to department or similar unit
\institute{Department of Computer Science}

% Date can be anything.
% Comment out or use \today for today's date.
\date{Conf'2017}

\begin{document}
\maketitle

\begin{frame}{Outline}
    \begin{center}
        \tableofcontents[hideallsubsections]
    \end{center}
\end{frame}

%%-------------------%%
%% MAIN MATTER START %%
%%-------------------%%

\section{Section 1}

\begin{frame}{A frame with a super long title that will not fit on one line at all}
    Some frame contents\footnote{With a footnote, that is down here}
\end{frame}

\section{Section 2}

\begin{frame}{Example of an itemized list}
    \begin{itemize}
        \item Level 1
        \begin{itemize}
            \item Level 2
            \begin{itemize}
                \item Level 3
            \end{itemize}
        \end{itemize}
    \end{itemize}
\end{frame}

% NB: subsection and subsubsections are not show on the TOC
\subsection{Subsection 2.1}

\begin{frame}{An enumerated list}
    \begin{enumerate}
        \item Level 1
        \begin{enumerate}
            \item Level 2
            \begin{enumerate}
                \item Level 3
            \end{enumerate}
        \end{enumerate}
    \end{enumerate}
\end{frame}

\subsubsection{Subsubsection 2.1.1}

\begin{frame}{A slide with lots of content}
    \lipsum[2]
\end{frame}

\begin{frame}[smalltitle]{A slide with a smaller title}
  \textit{Using the \texttt{smalltitle} option gives a smaller UH flame,
  leaving more space for your content...}

  \vspace{10pt}

  \lipsum[4]
\end{frame}

\begin{frame}[smalltitle]{A slide with a smaller title}
	\framesubtitle{And a subtitle}

  	\lipsum[4]
\end{frame}

\begin{frame}[fragile]{A slide with a subtitle}
  \framesubtitle{Concerning the use of subtitles}

  Use the \verb#\framesubtitle# command to add a subtitle
  underneath your frame's main title.

  \vfill

  Note also that we used \verb#fragile# on this slide: otherwise
  the \verb#\verb# command doesn't work.
\end{frame}

% NB: The [c] makes the frame vertically centered even if
% the default on first line of this file is 't'.
\begin{frame}[c]{A frame with blocks}
	% The blocks will be colored accoring to the faculty color scheme
	\begin{block}{Block with math}
		$\mathbf{X} = \{1, 2, 3\}$
	\end{block}
	\begin{block}{Block with text}
		Some text
	\end{block}
\end{frame}

\begin{frame}{Slide with references}
	\centering \cite{upper1974unsuccessful, didden2007multisite}
\end{frame}

\begin{frame}[allowframebreaks]{References}
	\bibliographystyle{abbrv} % bib formating
    \bibliography{bibliography} % bib file
\end{frame}

\end{document}
